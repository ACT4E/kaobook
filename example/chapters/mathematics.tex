\setchapterpreamble[u]{\margintoc}
\chapter{Mathematics and Boxes}

\section{Theorems}

\blindtext

\begin{definition}
Let $(X, d)$ be a metric space. A subset $U \subset X$ is an open set 
if, for any $x \in U$ there exists $r > 0$ such that $B(x, r) \subset 
U$. We call the topology associated to d the set $\tau\textsubscript{d}$ 
of all the open subsets of $(X, d).$
\end{definition}

\begin{theorem}
A finite intersection of open sets of (X, d) is an open set of (X, d), 
i.e $\tau\textsubscript{d}$ is closed under finite intersections. Any 
union of open sets of (X, d) is an open set of (X, d).
\end{theorem}

\begin{proposition}
A finite intersection of open sets of (X, d) is an open set of (X, d), 
i.e $\tau\textsubscript{d}$ is closed under finite intersections. Any 
union of open sets of (X, d) is an open set of (X, d).
\end{proposition}

\begin{lemma}
A finite intersection of open sets of (X, d) is an open set of (X, d), 
i.e $\tau\textsubscript{d}$ is closed under finite intersections. Any 
union of open sets of (X, d) is an open set of (X, d).
\end{lemma}

\begin{corollary}
A finite intersection of open sets of (X, d) is an open set of (X, d), 
i.e $\tau\textsubscript{d}$ is closed under finite intersections. Any 
union of open sets of (X, d) is an open set of (X, d).
\end{corollary}

\begin{proof}
The proof is left to the reader as a trivial exercise.
\end{proof}

\begin{definition}
Let $(X, d)$ be a metric space. A subset $U \subset X$ is an open set 
if, for any $x \in U$ there exists $r > 0$ such that $B(x, r) \subset 
U$. We call the topology associated to d the set $\tau\textsubscript{d}$ 
of all the open subsets of $(X, d).$
\end{definition}

\begin{example}
Let $(X, d)$ be a metric space. A subset $U \subset X$ is an open set 
if, for any $x \in U$ there exists $r > 0$ such that $B(x, r) \subset 
U$. We call the topology associated to d the set $\tau\textsubscript{d}$ 
of all the open subsets of $(X, d).$
\end{example}

\begin{examples}
\begin{subexample}
Simple example
\end{subexample}
\begin{subexample}
Simple example
\end{subexample}
\end{examples}

\begin{remark}
Let $(X, d)$ be a metric space. A subset $U \subset X$ is an open set 
if, for any $x \in U$ there exists $r > 0$ such that $B(x, r) \subset 
U$. We call the topology associated to d the set $\tau\textsubscript{d}$ 
of all the open subsets of $(X, d).$
\end{remark}

\begin{remark}
Integral $\int_{a}^{b} x^2 dx$ inline
\[\int_{a}^{b} x^2 dx\]
\end{remark}

\section{Boxes}

\blindtext

\begin{kaobox}[frametitle=Title of the box]
\blindtext
\end{kaobox}

\begin{tcbtheorem}{Fermat}{fermat}
test theorem
\end{tcbtheorem}
\begin{tcbproof}{Theorem \ref{thm:fermat}}{}
Easy.
\end{tcbproof}

%\begin{definition∗}[Inhomogeneous linear]
%\blindtext
%\end{definition∗}

\section{Experiments}

\blindtext

\marginnote{
	\begin{kaobox}[frametitle=title of margin note]
		Margin note inside a kaobox.\\
		(Actually, kaobox inside a marginnote!)
	\end{kaobox}
}

\blindtext

\begin{margintable}
	\captionsetup{type=table,position=above}
	\begin{kaobox}
		\caption{caption}
		\begin{tabular}{ |c|c|c|c| } 
			\hline
			col1 & col2 & col3 \\
			\hline
			\multirow{3}{4em}{Multiple row} & cell2 & cell3 \\ 
			& cell5 & cell6 \\ 
			& cell8 & cell9 \\ 
			\hline
		\end{tabular}
	\end{kaobox}
\end{margintable}
