\setchapterimage[6cm]{images/seaside}
\setchapterpreamble[u]{\margintoc}
\chapter{Page Layout}
\labch{layout}

\section{Headings}

So far, in this document I used two different styles for the chapter 
headings: one has the chapter name, a rule and, in the margin, the 
chapter number; the other has an image at the top of the page, and the 
chapter title is printed in a box (like for this chapter). There is one 
additional style, which I used only in the appendix 
(\refpage{appendix}); there, the chapter title is enclosed in two 
horizontal rules, and the chapter number (or letter, in the case of the 
appendix) is above it.\sidenote{To be honest, I do not think that mixing 
heading styles like this is a wise choice, but in this document I did 
only to show you how they look.}

Every book is unique, so it makes sense to have different styles from 
which to choose. Actually, it would be awesome if whenever a 
\Class{kao}-user designs a new heading style, he or she added it to the 
three styles already present, so that it will be available for new users 
and new books.

The choice of the style is made simple by the \Command{setchapterstyle} 
command. It accepts one option, the name of the style, which can be: 
\enquote{plain}, \enquote{kao}, or \enquote{lines}.\sidenote{Plain is 
the default \LaTeX\xspace title style; the other ones are self 
explanatory.} If instead you want the image style, you have to use the 
command \Command{setchapterimage}, which accepts the path to the image 
as argument; you can also provide an optional parameter in square 
brackets to specify the height of the image.

Let us make some examples. In this book, I begin a normal chapter with 
the lines:

\begin{lstlisting}
\setchapterstyle{kao}
\setchapterpreamble[u]{\margintoc}
\chapter{Title of the Chapter}
\labch{title}
\end{lstlisting}

Once the chapter style is set, it holds until you change it. When I want 
to start a chapter with an image, I simply write:

\begin{lstlisting}
\setchapterimage[7cm]{path/to/image.png}
\setchapterpreamble[u]{\margintoc}
\chapter{Catchy Title}
\labch{catchy}
\end{lstlisting}

\section{Headers \& Footers}

\blindtext

\section{Table of Contents}

Another option that is activated by default changes the style of the 
table of contents. By default, there is an entry for everything: list of 
figures, list of tables, indices, glossaries and bibliographies. There 
are also entries for the table of contents itself (thanks to the 
\Command{setuptoc\{toc\}\{totoc\}} command). If you want entries for the 
glossaries as well, you can set the \Option{toc} option of the package 
\Package{glossaries}.\sidenote[-7mm][]{If you don't want all these 
things in the table of contents, pass the appropriate KOMA options to 
the class.}

By default, dispositions are numbered up to the section thanks to the 
command \Command{setcounter\{secnumdepth\}\{1\}}. The table of contents 
can be modified through the package \Package{etoc}, which is loaded 
because it is needed for the margintocs, or the more traditional 
\Package{tocbase}. The sidenotes are numbered on a per-chapter basis, 
with the \Package{chngcntr} package; if you want to have only one 
counter for the whole document, check the provided \Path{style.sty} 
file.

\marginnote[1.5\parskip]{We also load \texttt{xcolor}.}

The sidenote counter is reset at every chapter, but you can change that 
with the \verb|\counterwithout| command.

 The space between the figure and the text can be specified with the 
following commands:

\begin{lstlisting}[style=kaolstplain]
\renewcommand\FBaskip{4pt}
\renewcommand\FBbskip{4pt}
\end{lstlisting}

