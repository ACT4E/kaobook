\setchapterstyle{kao}
%\setchapterpreamble[u]{\margintoc}
\chapter{References}
\labch{references}

\section{Citations}

\index{citations}
To cite someone \sidecite{Visscher2008,James2013} is very simple: just 
use the \Command{sidecite}\index{\Command{sidecite}} command. It does 
not have an offset argument yet, but it probably will in the future. 
This command supports multiple entries, as you can see, and by default 
it prints the reference on the margin as well as adding it to the 
bibliography at the end of the document. Note that the citations have 
nothing to do with the text,\sidecite{James2013} but they are completely 
random as they only serve the purpose to illustrate the feature.

For this setup I wrote a separate package, \Package{kaobiblio}, which 
you can find in the \Package{styles} directory and include in your main 
tex file. This package accepts all the options that you can pass to 
\Package{biblatex}, and actually it passes them to \Package{biblatex} 
under the hood. Moreover, it also defines some commands, like 
\Command{sidecite}, and environments that can be used within a 
\Class{kao} book.\sidenote{For this reason you should always use 
\Package{kaobiblio} instead of \Package{biblatex}, but the syntax and 
the options are exactly the same.}

If you want to use \Package{bibtex} instead of \Package{biblatex}, pass 
the option \Option{backend=bibtex} to \Package{kaobiblio}.

As you have seen, the \Command{sidecite} command will print a citation 
in the margin. However, this command would be useless without a way to 
customise the format of the citation, so the \Class{kaobook} provides 
also the \Command{formatmargincitation} command. By \enquote{renewing} 
that command, you can choose which items will be printed in the margins. 
The best way to understand how it works is to see the actual definition 
of this command.

\begin{lstlisting}[style=kaolstplain,linewidth=1.5\textwidth]
\newcommand{\formatmargincitation}[1]{%
	\parencite{#1}: \citeauthor*{#1} (\citeyear{#1}), \citetitle{#1}%
}
\end{lstlisting}

Thus, the \Command{formatmargincitation} accepts one parameter, which is 
the citation key, and prints the parencite followed by a colon, then the 
author, then the year (in brackets), and finally the 
title.\sidecite{Battle2014} Now, suppose that you wish the margin 
citation to display the year and the author, followed by the title, and 
finally a fixed arbitrary string; you would add to your document:

\begin{lstlisting}[style=kaolstplain,linewidth=1.5\textwidth]
\renewcommand{\formatmargincitation}[1]{%
	\citeyear{#1}, \citeauthor*{#1}: \citetitle{#1}; very interesting!%
}
\end{lstlisting}

\renewcommand{\formatmargincitation}[1]{%
	\citeyear{#1}, \citeauthor*{#1}: \citetitle{#1}; very interesting!%
}

The above code results in citations that look like the 
following.\sidecite{Zou2005} Of course, changing the format is most 
useful when you also change the default bibliography style. For 
instance, if you want to use the \enquote{philosophy-modern} style for 
your bibliography, you might have something like this in the preamble:

\begin{lstlisting}[style=kaolstplain,linewidth=1.5\textwidth]
\usepackage[style=philosophy-modern]{styles/kaobiblio}
\renewcommand{\formatmargincitation}[1]{%
	\sdcite{#1}%
}
\addbibresource{main.bib}
\end{lstlisting}

\renewcommand{\formatmargincitation}[1]{%
	\parencite{#1}: \citeauthor*{#1} (\citeyear{#1}), \citetitle{#1}%
}

The commands like \Command{citeyear}, \Command{parencite}
and \Command{sdcite} are just examples. A full
reference of the available commands can be found in this
\href{http://tug.ctan.org/info/biblatex-cheatsheet/biblatex-cheatsheet.pdf}{cheatsheet},
under the \enquote{Citations} section.

Finally, to compile a document containing citations, you need to use an 
external tool, which for this class is biber. You need to run the 
following (assuming that your tex file is called main.tex):

\begin{lstlisting}[style=kaolstplain]
$ pdflatex main
$ biber main
$ pdflatex main
\end{lstlisting}

\section{Glossaries and Indices}

\index{glossary}
The \Class{kaobook} class loads the packages \Package{glossaries} and 
\Package{imakeidx}, with which you can add glossaries and indices to 
your book. For instance, I previously defined some glossary entries and 
now I am going to use them, like this: \gls{computer}. 
\Package{glossaries} also allows you to use acronyms, like the 
following: this is the full version, \acrfull{fpsLabel}, and this is the 
short one \acrshort{fpsLabel}. These entries will appear in the glossary 
in the backmatter.

Unless you use \href{https://www.overleaf.com}{Overleaf} or some other 
fancy IDE for \LaTeX, you need to run an external command from your 
terminal in order to compile a document with a glossary. In particular, 
the commands required are:\sidenote{These are the commands you would run 
in a UNIX system, but see also \nrefsec{compiling}; I have no idea about 
how it works in Windows.}

\begin{lstlisting}[style=kaolstplain]
$ pdflatex main
$ makeglossaries main
$ pdflatex main
\end{lstlisting}

Note that you need not run \texttt{makeglossaries} every time you 
compile your document, but only when you change the glossary entries.

\index{index}
To create an index, you need to insert the command 
\lstinline|\index{subject}| whenever you are talking about 
\enquote{subject} in the text. For instance, at the start of this 
paragraph I would write \lstinline|index{index}|, and an entry would be 
added to the Index in the backmatter. Check it out!

\marginnote[2mm]{In theory, you would need to run an external command 
for the index as well, but luckily the package we suggested, 
	\Package{imakeidx}, can compile the index automatically.}

\index{nomenclature}
A nomenclature is just a special kind of index; you can find one at the end of
this book. To insert a nomenclature, we use the package \Package{nomencl} and
add the terms with the command \Command{nomenclature}. We put then a
\Command{printnomenclature} where we want it to appear.

Also with this package we need to run an external command to compile the 
document, otherwise the nomenclature will not appear:

\begin{lstlisting}[style=kaolstplain]
$ pdflatex main
$ makeindex main.nlo -s nomencl.ist -o main.nls
$ pdflatex main
\end{lstlisting}

These packages are all loaded in 
\href{style/packages.sty}{packages.sty}, one of the files that come with 
this class. However, the configuration of the elements is best done in 
the main.tex file, since each book will have different entries and 
styles.

Note that the \Package{nomencl} package caused problems when the 
document was compiled, so, to make a long story short, I had to prevent 
\Package{scrhack} to load the hack-file for \Package{nomencl}. When 
compiling the document on Overleaf, however, this problem seem to 
vanish.

\marginnote[-19mm]{This brief section was by no means a complete 
reference on the subject, therefore you should consult the documentation 
of the above package to gain a full understanding of how they work.}

\section{Hyperreferences}
\labsec{hyprefs}

\index{hyperreferences}
Together with this class we provide a handy package to help you 
referencing the same elements always in the same way, for consistency 
across the book. First, you can label each element with a specific 
command. For instance, should you want to label a chapter, you would put 
\lstinline|\labch{chapter-title}| right after the \Command{chapter} 
directive. This is just a convienence, because \Command{labch} is 
actually just an alias to \lstinline|\label{ch:chapter-title}|, so it 
spares you the writing of \enquote{ch:}. We defined similar commands for 
many typically labeled elements, including:

\begin{multicols}{2}
\setlength{\columnseprule}{0pt}
\begin{itemize}
	\item Page: \Command{labpage}
	\item Part: \Command{labpart}
	\item Chapter: \Command{labch}
	\item Section: \Command{labsec}
	\item Figure: \Command{labfig}
	\item Table: \Command{labtab}
	\item Definition: \Command{labdef}
	\item Assumption: \Command{labassum}
	\item Theorem: \Command{labthm}
	\item Proposition: \Command{labprop}
	\item Lemma: \Command{lablemma}
	\item Remark: \Command{labremark}
	\item Example: \Command{labexample}
	\item Exercise: \Command{labexercise}
\end{itemize}
\end{multicols}

Of course, we have similar commands for referencing those elements. 
However, since the style of the reference should depend on the context, 
we provide different commands to reference the same thing. For instance, 
in some occasions you may want to reference the chapter by name, but 
other times you want to reference it only by number. In general, there 
are four reference style, which we call plain, vario, name, and full.

The plain style references only by number. It is accessed, for chapters, 
with \lstinline|\refch{chapter-title}| (for other elements, the syntax 
is analogous). Such a reference results in: \refch{references}.

The vario and name styles rest upon the \Package{varioref} package. 
Their syntax is \lstinline|\vrefch{chapter-title}| and 
\lstinline|\nrefch{chapter-title}|, and they result in: 
\vrefch{references}, for the vario style, and: \nrefch{references}, for 
the name style. As you can see, the page is referenced in 
\Package{varioref} style.

The full style references everything. You can use it with 
\lstinline|\frefch{chapter-title}| and it looks like this: 
\frefch{references}.

Of course, all the other elements have similar commands (\eg for parts 
you would use \lstinline|\vrefpart{part-title}| or something like that). 
However, not all elements implement all the four styles. The commands 
provided should be enough, but if you want to see what is available or 
to add the missing ones, have a look at the 
\href{styles/kaorefs.sty}{attached package}.

In order to have access to all these features, the \Package{kaorefs} 
should be loaded in the preamble of your document. It should be loaded 
last, or at least after \Package{babel} (or \Package{polyglossia}) and 
\Package{plaintheorems} (or \Package{mdftheorems}). Options can be 
passed to it like to any other package; in particular, it is possible to 
specify the language of the captions. For instance, if you specify 
\enquote{italian} as an option, instead of \enquote{Chapter} it will be 
printed \enquote{Capitolo}, the Italian analog. If you know other 
languages, you are welcome to contribute the translations of these 
captions! Feel free to contact the author of the class for further 
details. 

The \Package{kaorefs} package also include \Package{cleveref}, so it is 
possible to use \Command{cref} in addition to all the previously 
described referencing commands.

\section{A Final Note on Compilation}
\labsec{compiling}

Probably the easiest way to compile a latex document is with the 
\Package{latexmk} script, as it can take care of everything, if properly 
configured, from the bibliography to the glossary. The command to issue, 
in general, is:

\begin{lstlisting}
latexmk [latexmk_options] [filename ...]
\end{lstlisting}

\Package{latexmk} can be extensively configurated (see 
\url{https://mg.readthedocs.io/latexmk.html}). For convenience, I print 
here an example configuration that would cover all the steps described 
above.

\begin{lstlisting}
# By default compile only the file called 'main.tex'
@default_files = ('main.tex');

# Compile the glossary and acronyms list (package 'glossaries')
add_cus_dep( 'acn', 'acr', 0, 'makeglossaries' );
add_cus_dep( 'glo', 'gls', 0, 'makeglossaries' );
$clean_ext .= " acr acn alg glo gls glg";
sub makeglossaries {
   my ($base_name, $path) = fileparse( $_[0] );
   pushd $path;
   my $return = system "makeglossaries", $base_name;
   popd;
   return $return;
}

# Compile the nomenclature (package 'nomencl')
add_cus_dep( 'nlo', 'nls', 0, 'makenlo2nls' );
sub makenlo2nls {
    system( "makeindex -s nomencl.ist -o \"$_[0].nls\" \"$_[0].nlo\"" );
}
\end{lstlisting}
