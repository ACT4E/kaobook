\newglossaryentry{computer}
{
  name=computer,
  description={is a programmable machine that receives input,
               stores and manipulates data, and provides
               output in a useful format}
}
\newacronym[longplural={Frames per Second}]{fpsLabel}{FPS}{Frame per Second}

\renewcommand*{\chapterformat}
{
  \enskip\mbox{\scalebox{4}{\thechapter\autodot}}
}
\renewcommand\chapterlinesformat[3]
{
  \parbox[b]{\textwidth}{\hrulefill#2}\par%
  #3%\par\bigskip
  \parbox[b]{\textwidth+\marginparsep+\marginparwidth}{\hrulefill}%
  %\hrule
}
\setchapterpreamble[u]{\margintoc}
\chapter{References}

\section{Citations}

To cite someone \sidecite{Visscher2008,James2013} is very simple: just 
use the \verb|\sidecite| command. It does not have an offset argument 
yet, but it probably will in the future. This command supports multiple 
entries, as you can see, and by default it prints the reference on the 
margin as well as adding it to the bibliography at the end of the 
document. For this setup I used biblatex but I think that workarounds 
are possible \sidecite{James2013}. Note that the citations have nothing 
to do with the text, they are completely random as they only serve the 
purpose to illustrate the feature.

\section{Glossaries and Indices}

I previously defined some glossary entries and now I am going to use 
them, for instance, like this: \gls{computer}. Since we are here, let us 
reference an acronym: this is the full version, \acrfull{fpsLabel}, and 
this is the short one \acrshort{fpsLabel}. These entries will appear in 
the glossary in the backmatter.

Now let us try the index\index{index}. I have just called 
\verb|\index{index}|, and an entry in the index has been added.
