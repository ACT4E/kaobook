\setchapterpreamble[u]{\margintoc[*-6]}
\setchapterstyle{plain}
\setchapterimage[6cm]{images/seaside}
\chapter{Page Layout}

\section{Headings}

\blindtext

\section{Headers \& Footers}

\blindtext

\section{Table of Contents}

Another option that is activated by default changes the style of the 
table of contents. By default, there is an entry for everything: list of 
figures, list of tables, indices, glossaries and bibliographies. There 
are also entries for the table of contents itself (thanks to the 
\Command{setuptoc\{toc\}\{totoc\}} command). If you want entries for the 
glossaries as well, you can set the \Option{toc} option of the package 
\Package{glossaries}.\sidenote[-7mm][]{If you don't want all these 
things in the table of contents, pass the appropriate KOMA options to 
the class.}

By default, dispositions are numbered up to the section thanks to the 
command \Command{setcounter\{secnumdepth\}\{1\}}. The table of contents 
can be modified through the package \Package{etoc}, which is loaded 
because it is needed for the margintocs, or the more traditional 
\Package{tocbase}. The sidenotes are numbered on a per-chapter basis, 
with the \Package{chngcntr} package; if you want to have only one 
counter for the whole document, check the provided \Path{style.sty} 
file.

\marginnote[1.5\parskip]{We also load \texttt{xcolor}.}

The sidenote counter is reset at every chapter, but you can change that 
with the \verb|\counterwithout| command.
