\setchapterstyle{kao}
%\setchapterpreamble[u]{\margintoc}
\chapter{References}

\section{Citations}

\index{citations}
To cite someone \sidecite{Visscher2008,James2013} is very simple: just 
use the \Command{sidecite}\index{\Command{sidecite}} command. It does 
not have an offset argument yet, but it probably will in the future. 
This command supports multiple entries, as you can see, and by default 
it prints the reference on the margin as well as adding it to the 
bibliography at the end of the document. For this setup I used biblatex 
but I think that workarounds are possible.\sidecite{James2013} Note that 
the citations have nothing to do with the text, they are completely 
random as they only serve the purpose to illustrate the feature.

\section{Glossaries and Indices}

\index{glossary}
If you load the packages \Package{glossaries} and \Package{imakeidx} you 
can add those fancy glossaries and indices to your book. For instance, I 
previously defined some glossary entries and now I am going to use them, 
like this: \gls{computer}. \Package{glossaries} also allows you to use 
acronyms, like the following: this is the full version, 
\acrfull{fpsLabel}, and this is the short one \acrshort{fpsLabel}. These 
entries will appear in the glossary in the backmatter.

Unless you use \href{https://www.overleaf.com}{Overleaf} or some other 
fancy IDE for \LaTeX, you need to run an external command from your 
terminal in order to compile a document with a glossary. In particular, 
the commands required are:\sidenote[-2mm][]{These are the commands you 
would run on a UNIX system; for Windows, have a look at the 
documentation.}

\begin{lstlisting}[style=kaolstplain]
$ pdflatex main.tex
$ makeglossaries main
$ pdflatex main.tex
\end{lstlisting}

Note that you need not run \texttt{makeglossaries} every time you 
compile your document, but only when you change the glossary entries.

\index{index}
To create an index, you need to insert the command 
\Command{index\{subject\}} whenever you are talking about 
\enquote{subject} in the text. For instance, at the start of this 
paragraph I would write \Command{index\{index\}}, and an entry would be 
added to the Index in the backmatter. Check it out!

\marginnote[2mm]{In theory, you would need to run an external command 
for the index as well, but luckily the package we suggested, 
	\Package{imakeidx}, can compile the index automatically.}

\index{nomenclature}
A nomenclature is just a special kind of index; you can find one at the 
end of this book. To insert a nomenclature, use the package 
\Package{nomencl} and add the terms with the command 
\Command{nomenclature}. Then, put a \Command{printnomenclature} where 
you want the nomenclature to appear.

Also with this package we need to run an external command to compile the 
document, otherwise the nomenclature will not appear:

\begin{lstlisting}[style=kaolstplain]
$ pdflatex main.tex
$ makeindex main.nlo -s nomencl.ist -o main.nls
$ pdflatex main.tex
\end{lstlisting}
