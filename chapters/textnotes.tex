%\renewcommand*{\chapterformat}
%{
  %\enskip\mbox{\scalebox{3.5}{\framebox{\thechapter\autodot}}}
%}
%\renewcommand\chapterlinesformat[3]
%{
  %\parbox[b]{\textwidth+\marginparsep+\marginparwidth}{
	%\parbox[b]{\textwidth}{#3}%
	%\parbox[b]{\marginparsep}{\hfill}%
	%\parbox[b]{\marginparwidth}{#2}%
  %}
  %%\hrule
%}
\setchapterpreamble[u]{\margintoc}
\chapter{Sidenotes and Marginnotes}

\section{Sidenotes}

To insert a sidenote, just enter the command 
\begin{verbatim}
\sidenote{Text of the note}
\end{verbatim}.
You can specify a mark\sidenote[O]{This sidenote has a special mark, a 
	big O!} with
\begin{verbatim}
\sidenote[mark]{Text}
\end{verbatim}
or you can specify an offset and a mark with
\begin{verbatim}
\sidenote[offset][mark]{Text}
\end{verbatim}
in which case the mark can be empty. If you want to know more, read the 
documentation of the \verb|snotez| package.

Sidenotes are handled through the \verb|snotez| package, which relies on 
the \verb|marginnote| package. The sidenote counter is reset at every 
chapter, but you can change that with the \verb|\counterwithout| 
command.

\section{Marginnotes}

\marginnote{The command for margin notes comes from the 
	\texttt{marginnote} package, but it has been redefined in order to 
	change the position of the optional offset argument, which now 
	precedes the text of the note, while in the original version it was 
	at the end. Check the \texttt{marginnote} package.}

This command is similar to the previous one. You can use it like so:
\begin{verbatim}
\marginnote[offset]{Text}
\end{verbatim}
where the offset argument can be left out.

We load the packages \verb|marginnote|, \verb|marginfix| and 
\verb|placeins|. Since \verb|snotez| uses \verb|marginnote|, what we say 
for marginnotes is also valid for sidenotes. The style of marginnotes 
and captions is the same, and the notes are shifted slightly upwards 
(\verb|\renewcommand{\marginnotevadjust}{3pt}|) in order to allineate 
them to the bottom of the line of text where the marginnote is issued.

The offset option can be either a (positive or negative) length or a 
multiple of \verb|\baselineskip|, \eg
\begin{verbatim}
\marginnote[-12pt]{Text} or \marginnote[*-3]{Text}
\end{verbatim}

\section{Footnotes}

Footnotes force the reader to constantly move from one area of the page 
to the other. Arguably marginnotes solve this issue, so you should not 
use footnotes. Nevertheless, for completeness, we have the standard 
command \verb|\footnote|, just in case you want to put a footnote once 
in a while.\footnote{And this is how they look like. Notice that there 
is a back reference to the text; pretty cool, uh?}

\section{Margintoc}

Since we are talking about margins, we introduce here the 
\verb|\margintoc| command, which accepts a parameter for the vertical 
offset, like so: \verb|\margintoc[offset]|. It can be used in any point 
of the document, but we think it makes sense to use it at the beginning 
of chapters or parts. In this document I put it in the chapter preamble, 
with this code:

\marginnote{The font used in the margintoc is the same as the one for 
	the chapter entries in the main table of contents at the beginning 
	of the document.}

\begin{verbatim}
	\setchapterpreamble[u]{\margintoc}
	\chapter{Chapter title}
\end{verbatim}
