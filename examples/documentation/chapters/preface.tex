\chapter*{Preface}
\addcontentsline{toc}{chapter}{Preface} % Add the preface to the table of contents as a chapter

I am of the opinion that every \LaTeX\xspace geek, at least once during 
his life, feels the need to create his or her own class: this is what 
happened to me and here is the result, which, however, should be seen as 
a work still in progress. Actually, this class is not completely 
original, but it is a blend of all the best ideas that I have found in a 
number of guides, tutorials, blogs and tex.stackexchange.com posts. In 
particular, the main ideas come from two sources:

\begin{itemize}
	\item \href{https://3d.bk.tudelft.nl/ken/en/}{Ken Arroyo Ohori}'s 
		\href{ttps://3d.bk.tudelft.nl/ken/en/nl/ken/en/2016/04/17/a-1.5-column-layout-in-latex.html}{Doctoral 
			Thesis}, which served, with the author's permission, as a 
		backbone for the implementation of this class;
	\item The 
		\href{https://github.com/Tufte-LaTeX/tufte-latex}{Tufte-Latex 
			Class}, which was a model for the style.
\end{itemize}

The first chapter of this book is introductive and covers the most 
essential features of the class. Next, there is a bunch of chapters 
devoted to all the commands and environments that you may use in writing 
a book; in particular, it will be explained how to add notes, figures 
and tables, and references. The second part deals with the page layout 
and design, as well as additional features like coloured boxes and 
theorem environments.

I started writing this class as an experiment, and as such it should be 
regarded. Since it has always been indended for my personal use, it may 
not be perfect but I find it quite satisfactory for the use I want to 
make of it. I share this work in the hope that someone might find here 
the inspiration for writing his or her own class.

\begin{flushright}
	\textit{Federico Marotta}
\end{flushright}
